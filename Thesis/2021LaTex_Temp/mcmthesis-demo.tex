%%
%% This is file `mcmthesis-demo.tex',
%% generated with the docstrip utility.
%%
%% The original source files were:
%%
%% mcmthesis.dtx  (with options: `demo')
%% 
%% -----------------------------------
%% 
%% This is a generated file.
%% 
%% Copyright (C)
%%       2010 -- 2015 by Zhaoli Wang
%%       2014 -- 2019 by Liam Huang
%%       2019 -- present by latexstudio.net
%% 
%% This work may be distributed and/or modified under the
%% conditions of the LaTeX Project Public License, either version 1.3
%% of this license or (at your option) any later version.
%% The latest version of this license is in
%%   http://www.latex-project.org/lppl.txt
%% and version 1.3 or later is part of all distributions of LaTeX
%% version 2005/12/01 or later.
%% 
%% This work has the LPPL maintenance status `maintained'.
%% 
%% The Current Maintainer of this work is latexstudio.net.
%% 
%%
%% This is file `mcmthesis-demo.tex',
%% generated with the docstrip utility.
%%
%% The original source files were:
%%
%% mcmthesis.dtx  (with options: `demo')
%%
%% -----------------------------------
%%
%% This is a generated file.
%%
%% Copyright (C)
%%       2010 -- 2015 by Zhaoli Wang
%%       2014 -- 2019 by Liam Huang
%%       2019 -- present by latexstudio.net
%%
%% This work may be distributed and/or modified under the
%% conditions of the LaTeX Project Public License, either version 1.3
%% of this license or (at your option) any later version.
%% The latest version of this license is in
%%   http://www.latex-project.org/lppl.txt
%% and version 1.3 or later is part of all distributions of LaTeX
%% version 2005/12/01 or later.
%%
%% This work has the LPPL maintenance status `maintained'.
%%
%% The Current Maintainer of this work is Liam Huang.
%%
\documentclass{mcmthesis}
\mcmsetup{CTeX = false,   % 使用 CTeX 套装时,设置为 true
        tcn = 2100079, problem = B,
        sheet = true, titleinsheet = true, keywordsinsheet = true,
        titlepage = true, abstract = true}
\usepackage{newtxtext}%\usepackage{palatino}
\usepackage{lipsum}
\title{Our Thesis for 2021 MCM Problem B}
\author{Muyuan Peng \quad Wei Zhao \quad Zhenyu Zhao}
\date{\today}
\begin{document}
\begin{abstract}
 

\begin{keywords}
keyword1; keyword2
\end{keywords}
\end{abstract}
\maketitle
%% Generate the Table of Contents, if it's needed.
%% \tableofcontents
%% \newpage
%%
%% Generate the Memorandum, if it's needed.
%% \memoto{\LaTeX{}studio}
%% \memofrom{Liam Huang}
%% \memosubject{Happy \TeX{}ing!}
%% \memodate{\today}
%% \logo{\LARGE I'm pretending to be a LOGO!}
%% \begin{memo}[Memorandum]
%%   \lipsum[1-3]
%% \end{memo}
%%
\tableofcontents
\newpage
\section{Overview}
\subsection{Background}


 The wildfires in Australia recently has drawn great focus. Frequent wildfires bring fire extinguishing system great pressure, thus the assistance
of automatic devices, like Unmanned Aerial Vehicle(UAV), 
radio and unmanned reconnaissance, should be taken into considerition. 
\subsection{Restatement of the Problem}

To help the firefighting system in East Victoria State, we are required
to solve several problems. The unmanned devices to respond to the bushfires can be divided to two kinds,
the SSA Thermal Imaging system and the Radio Repaeters, with each UAV carrying
one of them. To gurantee that the Safty Cofficient meets the satandard, we should
balance the number of SSA and RR carried by UAVs, in order to acquire best economic benifits.\\
Moreover, the prediction of the bushfires in the next decade in East Victotia is required
we will explain how to apply our model to respond to the bushfires.We are also required 
to point out where the UAVs with Radio Repaeters should be.

\begin{figure}[htbp]
  \centering
  \includegraphics[scale=0.7]{figures/Vict_Map.png}
  \caption{Topographical Map of Eastern Victoria}
  \label{Topographical Map of Eastern Victoria}
\end{figure}


\section{Notations}
\begin{itemize}
  \item \textbf{Unmanned Aerial Vehicle} $\rightarrow$ UAV
  \item \textbf{Radio Repeater} $\rightarrow$ RR
  \item \textbf{Rapid Bushfire Response} $\rightarrow$ RBR
  \item \textbf{Surveillance and Situational Awareness} $\rightarrow$ SSA
  \item \textbf{The total number of SSA and RR}$\rightarrow$ m
  \item \textbf{The firespot}$\rightarrow$ x
  \item \textbf{The fireregion}$\rightarrow$ y
  \item \textbf{Vector (x,y)}$\rightarrow$ $\alpha$ 
  \item \textbf{Danger Cofficient}$\rightarrow$ d
\end{itemize}
\newpage


\section{Hypothesis and Justifications}
\begin{enumerate}
  \item \textbf{The EOC will be set near the fire and will not move}
  \item \textbf{The Radio Repaeters'(RR) flights are always at their highest speed and will keep their location at the fires}
  \item \textbf{The wildfires are considered as discrete dots(Firepoints)} 
  \item \textbf{The UAV with SSA system must work within 5Km from the firepoint}
  \item \textbf{Ignore the side effects like Doppler Effecet and the influence of the winds}
\end{enumerate}

\section{Model Overview}
\subsection{Optimal numbers Model}
To design a purchasing configuration for SSA and RR, we need to introduce a \textbf{Danger Cofficient}
to evaluate whether the number of devices can gurantee that the firefightres are safe enough to work at
the firespots. Due to various kinds of goods in the problem, this is a threedimensional
heterogeneous container loading problem. We use the Three Space Division
Method and Monte Carlo Simulation to solve it.
\subsection{Programming on Routes of UAVs}

\section{Model Theory}

\section{Model Implementation and Results}

%\section{Sensitivity Analysis}

\section{Disscussion}

\section{Conclusion}
\section{}




\end{document}
%% 
%% This work consists of these files mcmthesis.dtx,
%%                                   figures/ and
%%                                   code/,
%% and the derived files             mcmthesis.cls,
%%                                   mcmthesis-demo.tex,
%%                                   README,
%%                                   LICENSE,
%%                                   mcmthesis.pdf and
%%                                   mcmthesis-demo.pdf.
%%
%% End of file `mcmthesis-demo.tex'.

